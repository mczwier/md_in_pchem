\documentclass[10pt]{article}
\usepackage[tuftesque]{mdpchem}

\title{MD Simulations of Ideal and Real Gases}
\author{Physical Chemistry}
\date{Fall 2017}

\begin{document}
\maketitle
\begin{fullwidth}
  \tableofcontents
\end{fullwidth}

\section{Introduction}
In this week's lab, we will be performing atomic simulations of ideal and real gases.
The purpose of today's exercise is to gain an intuitive understanding for the microscopic behavior of gases. 
To do so, we will perform classical simulations of a non-interacting argon-like gas, using Newton's laws to determine the paths of atoms in time.
We will continue our simulations of the dynamics of a gas by allowing the argon atoms to interact (approximately) as they would in reality.

\section{Procedure}

  \subsection{Running an ideal gas simulation and visualizing the results}
  \label{sec:ideal-gas}
  Follow the instructions of your professor to start a terminal and copy the simulation files you need into a directory in which you can work.
  When done, execute the following commands, which run a simulation of 100 ideal gas particles with the mass of argon at a temperature of 100~K.

\begin{enumerate}
  \item Run the following commands.
    The \filename{mdrun} step should take about half a minute.
\begin{fullwidth}
\begin{Verbatim}
gmx grompp -f md.mdp -c equil_100.gro -p ideal.top -o ideal_100.tpr
gmx mdrun -nt 1 -nb cpu -s ideal_100.tpr -c ideal_100.gro -e ideal_100.edr \
          -g ideal_100.log -o ideal_100.trr
\end{Verbatim}
\end{fullwidth}
  The first command prepares the simulation.
    The second command runs the simulation; this writes a number of files, including \filename{ideal_100.log} and \filename{ideal_100.trr}, which will be important shortly.
    If you are typing the second command on one line, omit the backslash.
    Each command prints a ``\acro{GROMACS} cool quote'' (for some value of \emph{cool}) if it completes successfully.
    For example,
 \begin{Verbatim}
 gcq#118: "Step On the Brakes" (2 Unlimited)
 \end{Verbatim}
    If you do not see a line that starts with \verb+gcq#+, then something has gone wrong; examine the terminal output for error messages and consult with your colleagues.


  \item Load the simulation into \acro{VMD} and watch it evolve.
    Do this by starting \acro{VMD}, then in the \selection{VMD Main} window, select \selection{File}, then \selection{New Molecule}.
    Click the \selection{Browse} button, navigate to the directory you performed the simulation in, then choose \filename{equil_100.gro}.
    After returning to the \selection{Molecule File Browser} window, click \selection{Load}.
    This loads the first frame (initial positions) of the simulation.
    Click \selection{Browse} again, then choose \filename{ideal_100.trr}.
    When in the \selection{Molecule File Browser} window, choose \selection{Load all at once}, then click \selection{Load}.
    This loads the remainder of the simulation.
    Close the \selection{Molecule File Browser} window.

  \item To see the motion of the atoms clearly, the display representation must be changed.
    In the \selection{VMD Main} window, click on the \selection{Graphics} menu, then \selection{Representations}.
    Change the \selection{Drawing Method} drop-down to \selection{VDW}.
    Now, watch the movie by clicking the play button in the lower right of the \selection{VMD Main} window.
    Write comments in your notebook describing the paths, speeds, and interactions of the atoms.
    When you are done, exit \acro{VMD} by choosing \selection{File}, then \selection{Quit} in the \selection{VMD Main} window.

  \item View the simulation log, which contains information about the energy of the simulation as it progresses.
    To do this, open \filename{ideal_100.log} in a text editor.
    Go to the very end, then up to find the section that looks like the following:
\begin{Verbatim}
	<======  ###############  ==>
	<====  A V E R A G E S  ====>
	<==  ###############  ======>

	Statistics over 1000001 steps using 10001 frames
\end{Verbatim}
    Write down the average energy, temperature, and pressure in your notebook.
    The average energy is given in the rather strange unit of kilojoules per mole of simulation copies.
    Since our system copy includes 100 argon-like atoms, a mole of simulation copies contain 100 moles of argon atoms.
    Thus, the reported energy is best interpreted as kilojoules per 100 moles of argon atoms.
    \item \label{item:ideal-questions}
    Answer the following questions in your lab notebook.
    You will also need to write these answers in your report.
    \begin{enumerate}
      \item Are the directions of travel of particles distributed isotropically?
      \item Do all particles travel with the same speed or with different speeds?
      \item Is your simulation consistent with the ideal gas law?
      To answer this, you will need to know that this simulation is contained in a cube $76.172\munit{\AA}$ on a side.
      \item Is your simulation consistent with $\avg{E_K} = (3/2) R T$?
    \end{enumerate}
\end{enumerate}

\subsection{Extending to other temperatures}
Repeat the above procedure and analysis for ideal gases at 200~K and 300~K, by substituting \codelit{_200} and \codelit{_300} for \codelit{_100}, as appropriate.
In addition, describe the differences among the three the simulation ``movies'' in your lab notebook.

%\clearpage
\subsection{Extending to interacting atoms}
\begin{enumerate}
  \item Run each of the following commands, pressing Return after each line.
    This sets up and runs the simulation (a 5~ns simulation in steps of 1~fs, with atomic positions recorded every 0.1~ps).
    This will take 2 -- 5 minutes.
    
\begin{fullwidth}    
\begin{Verbatim}
gmx grompp -f md_real.mdp -p argon.top -c equil_100.gro -o real_100.tpr
gmx mdrun -nt 1 -nb cpu -s real_100.tpr -c real_100.gro -e real_100.edr \
          -g real_100.log -o real_100.trr
\end{Verbatim}
\end{fullwidth}
    
    Again, if typing the second command into one line, omit the backslash.

  \item View the evolution of the simulation in VMD.
    Describe the differences from the ideal gas simulation.
    Look especially for collisions, scattering, and dimerization of argon atoms.

  \item Repeat the analysis of \autoref{sec:ideal-gas} item~\ref{item:ideal-questions}.

  \item Repeat the above for simulations at 200 and 300~K.

  \end{enumerate}

  \subsection{Analysis of interactions}
  The \emph{radial distribution function} $g(r)$ describes how likely it is to find two particles at a given separation $r$.
  Specifically, $g(r)$ describes the probability $P(r)$ that two atoms are separated by a distance $r$, relative to the probability $P\subword{ideal}(r)$ that two particles of an ideal gas are separated by a distance $r$:
    \[
    g(r) = \frac{P(r)}{P\subword{ideal}(r)}
  \]

  The radial distribution function can be calculated using the \filename{gmx rdf} command.
  For example:
  \begin{fullwidth}
\begin{Verbatim}
gmx rdf -s real_100.tpr -f real_100.trr -ref 2 -sel 2 -o rdf_real_100.xvg
\end{Verbatim}
    \end{fullwidth}
  This is not a trivial calculation, so it takes a minute or tw
  This program produces a file named according to the \codelit{-o} option; in the above example, the output file is named \filename{rdf_real_100.xvg}.
  This file can be plotted with your favorite plotting tool, such as Excel.
  If importing the file to Excel, set the import options to a delimited, space-separated file, treating consecutive delimiters as one.
  Perform the following analysis:
  \begin{enumerate}
    \item Plot $g(r)$ for all three real gas simulations and one ideal gas simulation.
    Sketch them in your notebook in addition to plotting them in Excel.
    \item Explain the physical significance of the following observations:
      \begin{enumerate}
        \item $g(r) \approx 1$ for the ideal gas simulation (forgiving how noisy $g(r)$ is at separations less than 10~Å).
        \item $\displaystyle \lim_{r \to \infty} = 1$ for all simulations.
        \item There is a maximum in $g(r)$ around 3.8~Å.
        \item Is there any variation in height at the maximum with respect to temperature?
          If so, what does this mean?
      \end{enumerate}
  
    \item We will (eventually) determine that probability $P$ is related to free energy $\increment A$ by
    \[
    \increment A = -\kB T \ln \frac{P}{P_0}
    \]
      where $P_0$ is a defined reference state probability and $\kB$ is Boltzmann's constant.
      Here, $P$ means probability, \emph{not} pressure.
      Use $g(r)$ to estimate the free energy of argon dimerization at the three different temperatures considered.
    This will require you to choose an unambiguous reference state $P_0$ and explain your choice.
      Is there a trend in $\increment A$ as temperature changes?
      Does this trend make sense in terms of the dynamics of the argon particles?


  \item It is possible to extract the pair potential $V(r)$ that describes interatomic interactions from the pair correlation function $g(r)$.
    The pair potential gives the potential energy of a pair of argon atoms at a separation of $r$.
    The relationship between $g(r)$ and $V(r)$ is (again, not proven here)
    \[
      g(r) = \exp\parens{-V(r)/k_B T}
    \]
    Using your $g(r)$ data, extract $V(r)$ for each of the real gas simulations.
    Sketch your results in your notebook.

  \item Quantum mechanics predicts that the long-distance tail of $V(r)$ is proportional to $r^{-6}$.
    Fit the long-distance tail of $V(r)$ for your 300~K simulation to a function $f(r) = C/r^6$, which should be possible by a suitable manipulation of your $V(r)$ data followed by a linear regression.
      Your value of $C$ should be quoted with an error estimate.

    \item The pair potential $V(r)$ can be related to the $a$ and $b$ parameters of the van der Waals equation.
      Specifically,
      \[
      a = \frac{2\pi N_A^2 C}{3\sigma^3} \quad \text{and} \quad b=\frac{2\pi \sigma^3 N_A}{3}
      \]
      where $C$ is the constant from your $1/r^6$ fit above, $\sigma$ is the internuclear separation where $V(r) = 0$, and $N_A$ is Avogadro's number.
      Calculate $a$ and $b$ using your previous results.
      Compare the values you obtain to the known values for argon, which you must look up.
      Use this comparison to comment on the accuracy of the simulations you just performed.

  \end{enumerate}
      
\section{Report}
Write a brief report answering the questions posed above, along with all descriptions requested along the way.
You will need to include the graphs you generated.
Prepare your report in \acro{PDF} (or, if necessary, \acro{docx}) format and e-mail it to me, along with your spreadsheet files or computer code and scans of your notebook pages.

\end{document}

%%% Local Variables:
%%% mode: latex
%%% TeX-master: t
%%% End:
